\beginsong{Baranek}[by={Kult},
                     index={Mój dom murem podzielony}]
\beginverse

Ech ci ludzie, to brudne świnie
Co napletli o mojej dziewczynie
Jakieś bzdury o jej nałogach
No to po prostu litość i trwoga
Tak to bywa gdy ktoś zazdrości
Kiedy brak mu własnej miłości
Plotki płodzi, mnie nie zaszkodzi żadne obce zło
Na mój sposób widzieć ją

\endverse
\beginverse

Na głowie kwietny ma wianek
W ręku zielony badylek
A przed nią bieży baranek
A nad nią lata motylek

\endverse
\beginverse

Krzywdę robią mojej panience
Opluć chcą ją, podli zboczeńcy
Utopić chcą ją w morzu zawiści
Paranoicy, podli sadyści
Utaplani w brudnej rozpuście
A na gębach fałszywy uśmiech
Byle zagnać do swego bagna, ale wara wam!
Ja ją przecież lepiej znam

\endverse
\beginverse

Na głowie...

\endverse
\beginverse

Znów widzieli ją z jakimś chłopem
Znów pojechała do Saint Tropez
Znów męczyła się, Boże drogi
Znów na jachtach myła podłogi
Tylko czemu ręce ma białe?
Chciałem zapytać - zapomniałem
Ciało kłoniąc, skinęła dłonią, wsparła skroń o skroń
Znów zapadłem w nią jak w toń

\endverse
\beginverse

Na głowie...

\endverse
\beginverse

Ech, dziewczyna pięknie się stara
Kosi pieniądz, ma Jaguara
Trudno pracę z miłością zgodzić
Rzadziej może do mnie przychodzić
Tylko pyta kryjąc rumieniec
Czemu patrzę jak potępieniec?
Czemu zgrzytam, kiedy się pyta - czy ma ładny biust?
Czemu toczę pianę z ust?

\endverse
\beginverse

Na głowie...

\endverse
\endsong