\beginsong{Sen Katarzyny II}[by={Jacek Kaczmarski},
                     index={Na smyczy trzymam filozofów Europy}]
\beginverse

Na smyczy trzymam filozofów Europy
Podparłam armią marmurowe Piotra stropy
Mam psy, sokoły, konie, kocham łów szalenie
A wokół same zające i jelenie
Pałace stawiam głowy ścinam
Kiedy mi przyjdzie na to chęć
Mam biografów, portrecistów
I jeszcze jedno pragnę mieć...

\endverse

\beginverse

Stój Katarzyno! koronę carów
Sen taki jak ten może ci z głowy zdjąć

\endverse

\beginverse

Kobietą jestem ponad miarę swoich czasów
Nie bawią mnie umizgi bladych lowelasów
Ich miękkich palców dotyk budzi obrzydzenie
Już wolę łowić zające i jelenie
Ze wstydu potem ten i ów
Rzekł o mnie: niewyżyta Niemra
I pod batogiem nago biegł
Po śniegu dookoła Kremla

\endverse

\beginverse

Stój Katarzyno...

\endverse

\beginverse

Kochanka trzeba mi takiego jak imperium
Co by mnie brał tak, jak ja daję: całą pełnią
Co by i władcy i poddańca był wcieleniem
By mi zastąpił zające i jelenie
Co by rozumiał tak jak ja
Ten głupi dwór rozdanych ról
I pośród pochylonych głów
Dawał mi rozkosz albo ból

\endverse

\beginverse

Stój Katarzyno! koronę carów
Sen taki jak ten może ci z głowy zdjąć
Gdyby się kiedyś kochanek taki znalazł...
Wiem, sama wiem! Kazałabym go ściąć!

\endverse

\endsong