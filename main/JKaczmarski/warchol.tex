\beginsong{Warchoł}[by={Jacek Kaczmarski},
                     index={Warchoł krzyczą nie zaprzeczam}]
\beginverse

Warchoł! - krzyczą - nie zaprzeczam
Tylko własnym prawom ufam
Wolna wola jest człowiecza
Pergaminów nie posłucha

\endverse
\beginverse

Boska ręka w tym, czy diabla
Szpetnie to, czy właśnie pięknie
Wola moja jest jak szabla
Nagniesz ją za mocno - pęknie

\endverse
\beginverse

A niewprawną puścisz dłonią
W pysk odbije stali siła
Tak się naucz robić bronią
By naturą swą służyła

\endverse
\beginverse

ref.
Sprawa ze mną - jak kraj ten stara
I jak zwykle on - byle jaka
Nie zrobili ze mnie janczara
Nie uczynią też i dworaka

\endverse
\beginverse

Wychwalali zasługi i cnoty	
Podsycali pochlebstwem wady	
A ja służyć nie mam ochoty	
Warchołowi nikt nie da rady!

\endverse
\beginverse

Lubię tany, pełne dzbany	
Sute stoły i tapczany	
Płeć nadobną - niesurową	
I od święta Boże Słowo	

\endverse
\beginverse

Lecz ni ksiądz, ni okowita	
Piekłem straszy, niebem nęci	
Ani żadna mnie kobita	
Wokół palca nie okręci	

\endverse
\beginverse

Mój ból głowy, moja skrucha	
Moje kiszki, moja franca	
Moja wreszcie groza ducha	
Gdy Kostucha rwie do tańca!	

\endverse
\beginverse

ref.
Sprawa ze mną...

Wychwalali zasługi...

\endverse
\beginverse

Jakbym ja był człowiek z wosku	
W rękach wodzów, niewiast, klechów	
Mógłbym ich zostawić troskom	
Cały ciężar moich grzechów	

\endverse
\beginverse

Ale znam tych stróżów mienia	
Sędziów sumień, prawdy zakon	
Spekulantów odkupienia	
Bo znam siebie - jako tako	

\endverse
\beginverse

Ulepiony i pokłuty	
Niepotrzebny będę więcej	
Rzucą w ogień dla pokuty	
I umyją po mnie ręce	

\endverse
\beginverse

ref.
Sprawa ze mną...

\endverse
\beginverse

Zły? - być może. Dobry? - a czemu?	
Nie tak wiele znów pychy we mnie	
Dajcie żyć po swojemu - grzesznemu	
A i świętym żyć będzie przyjemniej!	

\endverse
\beginverse

Zły? - być może. Dobry? - a czemu?	
Nie tak wiele znów pychy we mnie	
Dajcie żyć po swojemu - grzesznemu	
A i świętym żyć będzie przyjemniej!

\endverse

\endsong