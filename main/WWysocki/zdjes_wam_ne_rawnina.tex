\beginsong{Здесь вам не равнина}[by={Высоцкий Владимир},
                     index={Я не люблю}]
\beginverse

Здесь вам не равнина, здесь климат иной,
Идут лавины одна за одной,
И здесь за камнепадом ревёт камнепад.
И можно свернуть, обрыв обогнуть,
Но мы выбираем трудный путь,
Опасный, как военная тропа.
И можно свернуть, обрыв обогнуть,
Но мы выбираем трудный путь,
Опасный, как военная тропа.

\endverse
\beginverse

Кто здесь не бывал, кто не рисковал,
Тот сам себя не испытал,
Пусть даже внизу он звёзды хватал с небес.
Внизу не встретишь, как не тянись,
За всю свою счастливую жизнь
Десятой доли таких красот и чудес.
Внизу не встретишь, как не тянись,
За всю свою счастливую жизнь

\endverse
\beginverse

Десятой доли таких красот и чудес.

\endverse
\beginverse

Нет алых роз и траурных лент,
И не похож на монумент
Тот камень, что покой тебе подарил.
Как Вечным огнём сверкает днём
Вершина изумрудным льдом,
Которую ты так и не покорил.
Как Вечным огнём сверкает днём
Вершина изумрудным льдом,
Которую ты так и не покорил.

\endverse
\beginverse

И пусть говорят, да, пусть говорят,
Но нет, никто не гибнет зря,
Так лучше, чем от водки и от простуд.
Другие придут, сменив уют
На риск и непомерный труд,
Пройдут тобой не пройденный маршрут.
Другие придут, сменив уют
На риск и непомерный труд,
Пройдут тобой не пройденный маршрут.

\endverse
\beginverse

Отвесные стены, а ну, не зевай,
Ты здесь на везение не уповай,
В горах ненадежны
Ни камень, ни лёд, ни скала.
Надеемся только на крепость рук,
На руки друга и вбитый крюк,
И молимся, чтобы страховка не подвела.
Надеемся только на крепость рук,
На руки друга и вбитый крюк,
И молимся, чтобы страховка не подвела.

\endverse
\beginverse

Мы рубим ступени, ни шагу назад,
И от напряженья колени дрожат,
И сердце готово к вершине бежать из груди.
Весь мир на ладони, ты счастлив и нем,
И только немного завидуешь тем,
Другим, у которых вершина ещё впереди.
Весь мир на ладони, ты счастлив и нем,
И только немного завидуешь тем,
Другим, у которых вершина ещё впереди.

\endverse
\endsong