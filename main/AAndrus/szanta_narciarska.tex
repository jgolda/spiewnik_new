\beginsong{Szanta narciarska}[by={Artur Andrus},
                     index={Nazywali go marynarz}]
\beginverse

\[d]Nazywali \[C]go \[d]marynarz,
Bo \[F]opaskę \[G]miał na \[A]oku.
\[H7]Na każdym stoku \[F]dziewczyna,
\[F]Dziewczyna na \[E]każdym \[d]stoku.
\[d]Pochodzi \[C]spod \[d]Poznania,
\[F]Podobno umie \[G]wróżyć z \[A]kart.
\[H7]Panny rwie na \[F]wiązania,
\[F]Mężatki - \[E]na długość \[d]nart.

\endverse
\beginverse

\[A]Caryco mokrego \[d]śniegu
\[A]Ratrakiem płynę do Ciebie pod \[H7]prąd.
Hej!
\[H7]Dobrze, że stoisz na \[F]brzegu,
Bo ja \[F]właśnie \[E]schodzę na \[d]ląd.

\endverse
\beginverse

Nigdy \[d]się nie \[C]lękał \[d]biedy
I się \[F]nie \[G]przejmował \[A]jutrem.
A \[H7]jego ratrak był \[F]kiedyś
\[F]Zwyczajnym \[E]rybackim \[d]kutrem.
I \[d]woził \[C]dorsze i \[d]śledzie.
\[F]Zimą i \[G]latem, okrągły \[A]rok.
\[H7]Teraz jak nieraz \[F]przejedzie
\[F]Rybami \[E]czuć cały \[d]stok.

\endverse
\beginverse

Caryco mokrego śniegu...

\endverse
\beginverse

Wszyscy w ^porcie ^^odetchnęli.
Zwiał ^nim się ^zakończył ^sezon.
Jeszcze się ^tam jak żagiel ^bieli
Jego ^czarny ^^kombinezon.
^Odpłynął ^pod ^Ustrzyki
I ^przez ^kobiety ^wpadł w kłopoty.
^Forsę z polowań ^na orczyki
^Przehulał ^na ^antybiotyk.

\endverse
\beginverse

Caryco mokrego śniegu...

\endverse
\beginverse

Jeśli ^kiedyś ^go ^zobaczysz
Na ^ratraku w ^podłym ^świecie,
To ^powiedz mu, że w ^Karpaczu
^Czekają na ^niego ^dzieci.
I ^kiedy ^opuszcza ^statek,
^Żeby się ^znowu ^oddać złu,
^Każda z dwudziestu ^siedmiu matek
^Dzieciątku ^śpiewa do ^snu:

\endverse
\beginverse

Caryco mokrego śniegu... x2

\endverse
\endsong